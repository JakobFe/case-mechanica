\section{Transformatiematrices}
\begin{itemize}
\item De matrix in \eqref{eq:tm1} is de transformatiematrix om coördinaten van het bewegend $x'y'z'$-assenstelsel naar het wereldassenstelsel om te zetten. Dit bewegend assenstelsel heeft een draaing met een hoek $\alpha$ rond de x-as.
\begin{equation}
R^{x'y'z' \rightarrow xyz}=
\begin{bmatrix}
1			&			0			&			0		   \\
0			&\cos(\alpha)&-\sin(\alpha)\\
0			&\sin(\alpha)&\cos(\alpha) \\
\end{bmatrix}
\label{eq:tm1}
\end{equation}
\item Het $x''y''z''$-assenstelsel heeft dezelfde oriëntatie als het $x'y'z'$-assenstelsel. Hier is dus geen transformatiematrix vereist.
\item De matrix in \eqref{eq:tm2} is de transformatiematrix om coördinaten van het $x'''y'''z'''$-assenstelsel naar het $x''y''z''$-assenstelsel om te zetten. Dit eerste bewegend assenstelsel heeft een draaing met een hoek $\beta$ rond de y-as.
\begin{equation}
R^{x'''y'''z''' \rightarrow x''y''z''}=
\begin{bmatrix}
\cos(\beta)	&			0			&\sin(\beta)\\
0						&			1			&			0		 \\
-\sin(\beta)&			0			&\cos(\beta)\\
\end{bmatrix}
\label{eq:tm2}
\end{equation}
\item Het $x''''y''''z''''$-assenstelsel heeft dezelfde oriëntatie als het $x'''y'''z'''$-assenstelsel. Hier is dus geen transformatiematrix vereist.
\item De matrix in \eqref{eq:tm3} is de transformatiematrix om coördinaten van het $x'''y'''z'''$-assenstelsel of van het $x''''y''''z''''$-assenstelsel naar het wereldassenstelsel ($xyz$) om te zetten. Hierbij worden de coördinaten eerst omgezet naar het $x'y'z'$-assenstelsel en pas daarna naar het wereldassenstelsel. Dit kan met volgende matrixvermenigvuldiging.
\begin{equation}
\begin{split}
R^{x'''y'''z''' \rightarrow xyz}&=R^{x'y'z' \rightarrow xyz} \cdot R^{x'''y'''z''' \rightarrow x''y''z''}\\
&=
\begin{bmatrix}
1			&			0			&			0		   \\
0			&\cos(\alpha)&-\sin(\alpha)\\
0			&\sin(\alpha)&\cos(\alpha) \\
\end{bmatrix}
\cdot
\begin{bmatrix}
\cos(\beta)	&			0			&\sin(\beta)\\
0						&			1			&			0		 \\
-\sin(\beta)&			0			&\cos(\beta)\\
\end{bmatrix}\\
&=\begin{bmatrix}
\cos(\beta)							&			0				&\sin(\beta)\\
\sin(\alpha)\sin(\beta)	&\cos(\alpha)	&-\sin(\alpha)\cos(\beta)\\
-\cos(\alpha)\sin(\beta)&\sin(\alpha)	&\cos(\alpha)\cos(\beta)\\
\end{bmatrix}
\end{split}
\label{eq:tm3}
\end{equation}
\item De transpose van bovenstaande matrices geeft vanzelfsprekend de transformatiematrices om coördinaten in de omgekeerde richting te transformeren.
\end{itemize}