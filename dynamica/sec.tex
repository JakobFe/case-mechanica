\section{Dynamica}
\subsection{Opgave 1}
\subsubsection{Impuls}
De snelheid van het punt D die nodig is om de ogenblikkelijke impulsvector van het landingsgestel te berekenen, werd al berekend in \eqref{eq:kin4.6}.
\begin{equation}
\vec{p}_{l}^{'}=m_{l}\vec{v}_{d}^{'}=
\left[ \begin {array}{c} m_{l}\frac{1}{4}\,\omega_{i}\,l_{3}\\ \noalign{\medskip}m_{l}(v_{v}+\omega_{g}\, \left( l_{1}+\frac{3}{4}\,l_{4}) \right) \\ \noalign{\medskip}-m_{l}\frac{3}{4}\,\omega_{i}\,l_{4}\end {array} \right]
\label{eq:dyn1.1}
\end{equation}
\subsubsection{Verandering van impuls}
Ook de versnelling van het punt D is hier nodig en werd eerder al berekend in \eqref{eq:kin4.12}.
\begin{equation}
\frac{\mathrm{d}\vec{p}_{l}^{'}}{\mathrm{d}t}=m_{l}\vec{a}_{d}^{'}
=\left[ \begin {array}{c} m_{l}(-{\omega_{g}}^{2} \left( l_{1}+\frac{3}{4}\,l_{4} \right) +\frac{1}{4} \,\alpha_{i}\,l_{3}-\frac{3}{4}\,{\omega_{i}}^{2}l_{4})\\ \noalign{\medskip}m_{l}(a_{v}+\alpha_{g}\, \left( l_{1}+\frac{3}{4}\,l_{4} \right) +\frac{3}{4}\,\omega_{g}\,\omega_{i}\,l_{3})\\ \noalign{\medskip}m_{l}(-\frac{3}{4}\,\alpha_{i}\,l_{4}-\frac{1}{4}\,{\omega_{i}}^{2}l_{3})\end {array} \right]
\label{eq:dyn1.2}
\end{equation}
\subsection{Opgave 2}
\subsection{Opgave 3}
\subsection{Opgave 4}
\subsection{Opgave 5}