\section{Kinematica}
\subsection{Opgave 1}
\subsubsection{Rotatiesnelheid}
Om de ogenblikkelijke totale rotatiesnelheidsvector $\vec{\omega}_{tot}$ te berekenen zetten we eerst alle rotatiesnelheidsvectoren om naar het $x'y'z'$-assenstelsel. $\vec{\omega}_{g}$ en $\vec{\omega}_{i}$ staan hier al in, dus we moeten enkel $\vec{\omega}_{w}$ nog omzetten. 
\begin{equation}
\begin{split}
\vec{\omega}_{w}^{'}&=R^{x'''y'''z''' \rightarrow x'y'z'} \cdot \vec{\omega}_{w}^{'''}\\
&=\begin{bmatrix}
\cos(\beta)	&			0			&\sin(\beta)\\
0						&			1			&			0		 \\
-\sin(\beta)&			0			&\cos(\beta)\\
\end{bmatrix}
\cdot
\begin{bmatrix}
-\omega_{w}	\\
0						\\
0						\\
\end{bmatrix}\\
&=
\begin{bmatrix}
-\omega_{w}\cdot \cos(\beta)	\\
0						\\
\omega_{w}\cdot \sin(\beta)	\\
\end{bmatrix}
\end{split}
\label{eq:kin1.1}
\end{equation}
Als we al deze vectoren optellen krijgen we $\vec{\omega}_{tot}^{'}$\footnote{Notatie: het aantal accenten als superscript bij vectoren duidt op het assenstelsel waar de vector in is uitgedrukt.}
\begin{equation}
\begin{split}
\vec{\omega}_{tot}^{'}&=\vec{\omega}_{g}^{'} + \vec{\omega}_{i}^{'} + \vec{\omega}_{w}^{'}\\
&=\begin{bmatrix}
0						\\
0						\\
\omega_{g}	\\
\end{bmatrix}
+\begin{bmatrix}
0						\\
\omega_{i}	\\
0						\\
\end{bmatrix}
+\begin{bmatrix}
-\omega_{w}\cdot \cos(\beta)	\\
0						\\
\omega_{w}\cdot \sin(\beta)	\\
\end{bmatrix}\\
&=\begin{bmatrix}
-\omega_{w}\cdot \cos(\beta)	\\
\omega_{i}						\\
\omega_{g}+\omega_{w}\cdot \sin(\beta)	\\
\end{bmatrix}
\end{split}
\label{eq:kin1.2}
\end{equation}
Na omvorming naar het wereldassenstelsel krijgen we de ogenblikkelijke totale rotatiesnelheidsvector $\vec{\omega}_{tot}$.
\begin{equation}
\begin{split}
\vec{\omega}_{tot}&=R^{x'y'z' \rightarrow xyz} \cdot \vec{\omega}_{tot}^{'}\\
&=\begin{bmatrix}
1			&			0			&			0		   \\
0			&\cos(\alpha)&-\sin(\alpha)\\
0			&\sin(\alpha)&\cos(\alpha) \\
\end{bmatrix}
\cdot
\begin{bmatrix}
-\omega_{w}\cdot \cos(\beta)	\\
\omega_{i}						\\
\omega_{g}+\omega_{w}\cdot \sin(\beta)	\\
\end{bmatrix}\\
&=
\begin{bmatrix}
-\omega_{w}\cdot \cos(\beta)	\\
\omega_{i}\cos(\alpha)-\sin(\alpha)(\omega_{g}+\omega_{w}\sin(\beta))\\
\omega_{i}\sin(\alpha)+\cos(\alpha)(\omega_{g}+\omega_{w}\sin(\beta))\\
\end{bmatrix}
\end{split}
\label{eq:kin1.3}
\end{equation}
\newpage
\subsubsection{Rotatieversnelling}
Om de ogenblikkelijke totale rotatieversnellingsvector $\vec{\alpha}_{tot}$ te berekenen maken we gebruik van formule \eqref{eq:kin1.4}.
\begin{equation}
\vec{\alpha} = \sum_{i=1}^{k}{\frac{\mathrm{d}\omega_{i}}{\mathrm{d}t}\vec{e}_{\omega_{i}}}+\sum_{i=1}^{k}{\vec{\omega}_{i}\frac{\mathrm{d}\vec{e}_{\omega_{i}}}{\mathrm{d}t}}  
\label{eq:kin1.4}
\end{equation}
met:
\begin{equation}
\vec{\omega}_{i}\frac{\mathrm{d}\vec{e}_{\omega_{i}}}{\mathrm{d}t}=\sum_{j=1}^{i-1}{\vec{\omega}_{j}\times\vec{\omega}_{i}}  
\label{eq:kin1.5}
\end{equation}
Concreet wordt dit:
\begin{equation}
\vec{\alpha}_{tot} = \vec{\alpha}_{g} + \vec{\alpha}_{i} + \vec{\alpha}_{w} + \vec{\omega}_{g} \times \vec{\omega}_{i} + \vec{\omega}_{g} \times \vec{\omega}_{w} + \vec{\omega}_{i} \times \vec{\omega}_{w}
\label{eq:kin1.6}
\end{equation}
$\vec{\omega}_{g} \times \vec{\omega}_{w}$ en $\vec{\omega}_{i} \times \vec{\omega}_{w}$ zijn termen die duiden op de verandering van oriëntatie van $\vec{\omega}_{w}$. Zijn oriëntatie in namelijk is afhankelijk van $\vec{\omega}_{g}$ en $\vec{\omega}_{i}$. $\vec{\omega}_{g} \times \vec{\omega}_{i}$ is om dezelfde reden toegevoegd.

$\vec{\alpha}_{g}$, $\vec{\alpha}_{i}$, $\vec{\omega}_{g}$ en $\vec{\omega}_{i}$ zijn reeds uitgedrukt in het $x'y'z'$-assenstelsel. $\vec{\omega}_{w}$ werd al in \eqref{eq:kin1.1} naar dit assenstelsel getransformeerd. Hieronder wordt $\vec{\alpha}_{w}$ getransformeerd.
\begin{equation}
\begin{split}
\vec{\alpha}_{w}^{'}&=R^{x'''y'''z''' \rightarrow x'y'z'} \cdot \vec{\alpha}_{w}^{'''}\\
&=\begin{bmatrix}
\cos(\beta)	&			0			&\sin(\beta)\\
0						&			1			&			0		 \\
-\sin(\beta)&			0			&\cos(\beta)\\
\end{bmatrix}
\cdot
\begin{bmatrix}
\alpha_{w}	\\
0						\\
0						\\
\end{bmatrix}\\
&=
\begin{bmatrix}
\alpha_{w}\cdot \cos(\beta)	\\
0						\\
-\alpha_{w}\cdot \sin(\beta)	\\
\end{bmatrix}
\end{split}
\label{eq:kin1.7}
\end{equation}
Uitgewerkt wordt $\vec{\alpha}_{tot}$:
\begin{equation}
\begin{split}
\vec{\alpha}_{tot}^{'} &= \vec{\alpha}_{g}^{'} + \vec{\alpha}_{i}^{'} + \vec{\alpha}_{w}^{'} + \vec{\omega}_{g}^{'} \times \vec{\omega}_{i}^{'} + \vec{\omega}_{g}^{'} \times \vec{\omega}_{w}^{'} + \vec{\omega}_{i}^{'} \times \vec{\omega}_{w}^{'}\\
&=\begin{bmatrix}
0						\\
0						\\
\alpha_{g}	\\
\end{bmatrix}
+\begin{bmatrix}
0						\\
\alpha_{i}	\\
0						\\
\end{bmatrix}
+\begin{bmatrix}
\alpha_{w}\cdot \cos(\beta)	\\
0						\\
-\alpha_{w}\cdot \sin(\beta)	\\
\end{bmatrix}
+\begin{vmatrix}
e_{x}&e_{y}&e_{z}\\
0&0&\omega_{g}\\
0&\omega_{i}&0\\
\end{vmatrix}\\
&+\begin{vmatrix}
e_{x}&e_{y}&e_{z}\\
0&0&\omega_{g}\\
-\omega_{w}\cdot \cos(\beta)&0&\omega_{w}\cdot \sin(\beta)\\
\end{vmatrix}
+\begin{vmatrix}
e_{x}&e_{y}&e_{z}\\
0&\omega_{i}&0\\
-\omega_{w}\cdot \cos(\beta)&0&\omega_{w}\cdot \sin(\beta)\\
\end{vmatrix}\\
&=\begin{bmatrix}
\alpha_{w}\cos(\beta)-\omega_{i}\omega_{g}+\omega_{i}\omega_{w}\sin{\beta}\\
\alpha_{i}-\omega_{g}\omega_{w}\cos{\beta}\\
\alpha_{g}-	\alpha_{w} \sin(\beta)+\omega_{i}\omega_{w}\cos{\beta}\\
\end{bmatrix}
\label{eq:kin1.8}
\end{split}
\end{equation}
\begin{equation}
\begin{split}
\vec{\alpha}_{tot}&=R^{x'y'z' \rightarrow xyz} \cdot \vec{\alpha}_{tot}^{'}\\
&=\begin{bmatrix}
1			&			0			&			0		   \\
0			&\cos(\alpha)&-\sin(\alpha)\\
0			&\sin(\alpha)&\cos(\alpha) \\
\end{bmatrix}
\cdot
\begin{bmatrix}
-\alpha_{w}\cos(\beta)-\omega_{i}\omega_{g}+\omega_{i}\omega_{w}\sin{\beta}\\
\alpha_{i}-\omega_{g}\omega_{w}\cos{\beta}\\
\alpha_{g}+	\alpha_{w} \sin(\beta)+\omega_{i}\omega_{w}\cos{\beta}\\
\end{bmatrix}
\end{split}
\label{eq:kin1.9}
\end{equation}
De uitwerking wordt hier achterwege gelaten wegens te complex.
\newpage
\subsection{Opgave 2}
\subsubsection{Snelheid}
Om de ogenblikkelijke snelheid $\vec{v}_{c}$ van het punt C te berekenen hebben wij voor een combinatie van methodes gekozen. Ten eerste is er een samengestelde beweging met het $x'y'z'$-assenstelsel als bewegend assenstelsel. Hierdoor heeft de sleepsnelheid in \eqref{eq:kin2.6} een component veroorzaakt door de translatie van het vliegtuig met snelheid $\vec{v}_{v}$ en een component veroorzaakt door de rotatie van het vliegtuig met rotatiesnelheid $\vec{\omega}_{g}$. Ten tweede maken we gebruik van de methode van som van rotaties om de relatieve snelheid in \eqref{eq:kin2.7} van de eerder vermelde samengestelde beweging te berekenen. We moeten eerst echter de afstandsvectoren berekenen.
\begin{equation}
\vec{r}_{AB}^{'}=
\begin{bmatrix}
l_{1}\\
0\\
l_{2}\\
\end{bmatrix}
\label{eq:kin2.3}
\end{equation}
\begin{equation}
\vec{r}_{BC}^{'}=
\begin{bmatrix}
-(l_{3}\sin{\beta}+l_{4}\cos{\beta})\\
0\\
-l_{3}\cos{\beta}+l_{4}\sin{\beta}\\
\end{bmatrix}
\label{eq:kin2.4}
\end{equation}
\begin{equation}
\vec{r}_{AC}^{'}=
\begin{bmatrix}
l_{1}-(l_{3}\sin{\beta}+l_{4}\cos{\beta})\\
0\\
l_{2}-l_{3}\cos{\beta}+l_{4}\sin{\beta}\\
\end{bmatrix}
\label{eq:kin2.5}
\end{equation}
De sleepsnelheid uitgedrukt in het $x'y'z'$-assenstelsel.
\begin{equation}
\begin{split}
\vec{v}_{sleep}^{'}&=\vec{v}_{v}^{'}+\vec{\omega}_{g}^{'}\times\vec{r}_{AC}^{'}\\
&=\begin{bmatrix}
0						\\
v_{v}	\\
0						\\
\end{bmatrix}
+\begin{vmatrix}
e_{x}&e_{y}&e_{z}\\
0&0&\omega_{g}\\
l_{1}-(l_{3}\sin{\beta}+l_{4}\cos{\beta})&0&l_{2}-l_{3}\cos{\beta}+l_{4}\sin{\beta}\\
\end{vmatrix}\\
&=\begin{bmatrix}
0						\\
v_{v}+\omega_{g}\,\left( l_{1}-\cos \left( \beta \right) l_{4}-\sin \left( \beta \right) l_{3} \right)\\
0						\\
\end{bmatrix}
\end{split}
\label{eq:kin2.6}
\end{equation}
Ook de relatieve snelheid wordt in het $x'y'z'$-assenstelsel uitgedrukt. $\vec{\omega}_{w}$ is niet opgenomen in de formule omdat deze geen bijdrage bij de snelheid van C levert.
\begin{equation}
\begin{split}
\vec{v}_{rel}^{'}&=\vec{\omega}_{i}^{'}\times\vec{r}_{BC}^{'}\\
&=\begin{vmatrix}
e_{x}&e_{y}&e_{z}\\
0&\omega_{i}&0\\
-(l_{3}\sin{\beta}+l_{4}\cos{\beta})&0&-l_{3}\cos{\beta}+l_{4}\sin{\beta}\\
\end{vmatrix}\\
&=\left[ \begin {array}{c} \omega_{i}\, \left( \sin \left( \beta
 \right) l_{4}-\cos \left( \beta \right) l_{3} \right)\\ 0\\-\omega_{i}\, \left( -\cos \left( \beta \right) l_{4}-\sin \left( \beta \right) l_{3} \right)\end {array} \right]
\end{split}
\label{eq:kin2.7}
\end{equation}
Na optelling van de sleepsnelheid en relatieve snelheid bekomen we \eqref{eq:kin2.8}.
\begin{equation}
\vec{v}_{c}^{'}=\left[ \begin {array}{c} \omega_{i}\, \left( \sin \left( \beta
 \right) l_{4}-\cos \left( \beta \right) l_{3} \right) 
\\ \noalign{\medskip}-\cos \left( \beta \right) l_{4}\,\omega_{g}-\sin
 \left( \beta \right) l_{3}\,\omega_{g}+l_{1}\,\omega_{g}+v_{v}
\\ \noalign{\medskip}\omega_{i}\, \left( \cos \left( \beta \right) l_{
4}+\sin \left( \beta \right) l_{3} \right) \end {array} \right]
\label{eq:kin2.8}
\end{equation}
\newpage
\subsubsection{Versnelling}
\label{sec:versnelling}
De versnelling wordt op dezelfde manier bepaald. We nemen opnieuw het $x'y'z'$-assenstelsel als bewegend assenstelsel voor de samengestelde beweging.
\begin{equation}
\vec{a}_{tot}^{'}=\vec{a}_{sleep}^{'}+\vec{a}_{rel}^{'}+\vec{a}_{comp}^{'}
\label{eq:kin2.9}
\end{equation}
met:
\begin{equation}
\begin{split}
\vec{a}_{sleep}^{'}&=\vec{a}_{v}^{'}+\vec{\alpha}_{g}^{'}\times\vec{r}_{AC}^{'}+\vec{\omega}_{g}^{'}\times(\vec{v}_{c}^{'}-\vec{v}_{a}^{'})\\
&=\begin{bmatrix}
0						\\
\alpha_{v}	\\
0						\\
\end{bmatrix}
+\begin{vmatrix}
e_{x}&e_{y}&e_{z}\\
0&0&\alpha_{g}\\
l_{1}-(l_{3}\sin{\beta}+l_{4}\cos{\beta})&0&l_{2}-l_{3}\cos{\beta}+l_{4}\sin{\beta}\\
\end{vmatrix}\\
&+\begin{vmatrix}
e_{x}&e_{y}&e_{z}\\
0&0&\omega_{g}\\
\omega_{i}\, \left( \sin \left( \beta \right) l_{4}-\cos \left( \beta \right) l_{3} \right)&\omega_{g}\, \left( l_{1}-\cos \left( \beta \right) l_{4}-\sin \left( \beta \right) l_{3} \right)&-\omega_{i}\, \left( -\cos \left( \beta \right) l_{4}-\sin \left( \beta \right) l_{3} \right)\\
\end{vmatrix}\\
&=\left[ \begin {array}{c} -{\omega_{g}}^{2} \left( l_{1}-\cos \left( \beta \right) l_{4}-\sin \left( \beta \right) l_{3} \right) 
\\a_{v}+\alpha_{g}\, \left( l_{1}-\cos \left( \beta \right) l_{4}-\sin \left( \beta \right) l_{3} \right) +\omega_{g}\,\omega_{i}\, \left( \sin \left( \beta \right) l_{4}-\cos \left( \beta \right) l_{3} \right) \\0\end {array} \right] 
\label{eq:kin2.10}
\end{split}
\end{equation}
\begin{equation}
\begin{split}
\vec{a}_{rel}^{'}&=\vec{\alpha}_{i}^{'}\times\vec{r}_{BC}^{'}+\vec{\omega}_{i}^{'}\times(\vec{v}_{c}^{'}-\vec{v}_{b}^{'})\\
&=\begin{vmatrix}
e_{x}&e_{y}&e_{z}\\
0&\alpha_{i}&0\\
-(l_{3}\sin{\beta}+l_{4}\cos{\beta})&0&-l_{3}\cos{\beta}+l_{4}\sin{\beta}\\
\end{vmatrix}\\
&+\begin{vmatrix}
e_{x}&e_{y}&e_{z}\\
0&\omega_{i}&0\\
\omega_{i}(\sin{\beta}l_{4}-\cos{\beta}l_{3})&-\omega_{g}(\cos{\beta}l_{4}+\sin{\beta}l_{3})&\omega_{i}(\cos{\beta}l_{4}+\sin{\beta}l_{3})\\
\end{vmatrix}\\
&=\left[ \begin {array}{c} \alpha_{i}\, \left( \sin \left( \beta
 \right) l_{4}-\cos \left( \beta \right) l_{3} \right) -{\omega_{i}}^{2} \left( -\cos \left( \beta \right) l_{4}-\sin \left( \beta \right) l_{3} \right) \\0\\ -\alpha_{i}\, \left( -\cos \left( \beta \right) l_{4}-\sin \left( \beta \right) l_{3} \right) -{\omega_{i}}^{2} \left( \sin \left( \beta \right) l_{4}-
\cos \left( \beta \right) l_{3} \right) \end {array} \right]
\end{split}
\label{eq:kin2.11}
\end{equation}
\begin{equation}
\begin{split}
\vec{a}_{comp}^{'}&=2\cdot(\vec{\omega}_{g}^{'}\times\vec{v}_{rel}^{'})\\
&=2\cdot
\begin{vmatrix}
e_{x}&e_{y}&e_{z}\\
0&0&\omega_{g}\\
\omega_{i}(\sin(\beta)l_{4}-\cos(\beta)l_{3})&0&\omega_{i}(\cos(\beta)l_{4}+\sin(\beta)l_{3})\\
\end{vmatrix}\\
&=\left[ \begin {array}{c} 0\\2\,\omega_{g}\,\omega_
{i}\, \left( \sin \left( \beta \right) l_{4}-\cos \left( \beta
 \right) l_{3} \right) \\ 0\end {array} \right]
\end{split}
\label{eq:kin2.12}
\end{equation}
Hieronder worden nog enkele termen van bovenstaande formules nader verklaard:
\begin{itemize}
\item In de derde term van de sleepsnelheid staat $\vec{v}_{c}^{'}-\vec{v}_{a}^{'}$. Hierbij is $\vec{v}_{c}^{'}$ de snelheid van het punt C dat in \eqref{eq:kin2.8} al berekend werd. $\vec{v}_{a}^{'}$ is de snelheid van het punt waar $vec{\omega}_{g}$ aangrijpt. In dit geval punt A. De snelheid van punt A is gelijk aan $\vec{v}_{v}^{'}$. 
\begin{equation}
\vec{v}_{c}^{'}-\vec{v}_{a}^{'}=\vec{v}_{c}^{'}-\vec{v}_{v}^{'}=
\left[ \begin {array}{c} \omega_{i}\, \left( \sin \left( \beta
 \right) l_{4}-\cos \left( \beta \right) l_{3} \right) 
\\ \noalign{\medskip}\omega_{g}\, \left( l_{1}-\cos \left( \beta
 \right) l_{4}-\sin \left( \beta \right) l_{3} \right) 
\\ \noalign{\medskip}\omega_{i}\, \left( \cos \left( \beta \right) l_{4}+\sin \left( \beta \right) l_{3} \right) \end {array} \right]
\label{eq:kin2.13}
\end{equation}
\item De tweede term van de relatieve snelheid bevat de snelheid van het punt B. Die wordt hieronder in detail uitgewerkt.
\begin{equation}
\begin{split}
\vec{v}_{b}^{'}&=\vec{v}_{v}^{'}+\vec{\omega}_{g}^{'}\times\vec{r}_{AB}^{'}\\
&=\begin{bmatrix}
0\\
v_{v}\\
0\\
\end{bmatrix}
+\begin{vmatrix}
e_{x}&e_{y}&e_{z}\\
0&0&\omega_{g}\\
l_{1}&0&l_{2}\\
\end{vmatrix}\\
&=\begin{bmatrix}
0\\
v_{v}+l_{1}\omega_{g}\\
0\\
\end{bmatrix}
\label{eq:kin2.14}
\end{split}
\end{equation}
\end{itemize}
Als we formule \eqref{eq:kin2.10},\eqref{eq:kin2.11} en \eqref{eq:kin2.12} in \eqref{eq:kin2.9} invullen krijgen we:
\begin{equation}
\vec{a}_{tot}^{'}=\left[ \begin {array}{c} -{\omega_{g}}^{2} \left( l_{1}-\cos \left( 
\beta \right) l_{4}-\sin \left( \beta \right) l_{3} \right) +\alpha_{i
}\, \left( \sin \left( \beta \right) l_{4}-\cos \left( \beta \right) l
_{3} \right) -{\omega_{i}}^{2} \left( -\cos \left( \beta \right) l_{4}
-\sin \left( \beta \right) l_{3} \right) \\ \noalign{\medskip}a_{v}+
\alpha_{g}\, \left( l_{1}-\cos \left( \beta \right) l_{4}-\sin \left( 
\beta \right) l_{3} \right) +3\,\omega_{g}\,\omega_{i}\, \left( \sin
 \left( \beta \right) l_{4}-\cos \left( \beta \right) l_{3} \right) 
\\ \noalign{\medskip}-\alpha_{i}\, \left( -\cos \left( \beta \right) l
_{4}-\sin \left( \beta \right) l_{3} \right) -{\omega_{i}}^{2} \left( 
\sin \left( \beta \right) l_{4}-\cos \left( \beta \right) l_{3}
 \right) \end {array} \right]
\label{eq:kin2.15}
\end{equation}
De volledige uitwerking naar het wereldassenstelsel zou te complex worden. Hieronder wordt dus louter de formule gegeven.
\begin{equation}
\vec{\alpha}_{tot}=R^{x'y'z' \rightarrow xyz}\cdot\vec{\alpha}_{tot}^{'}
\end{equation}
\newpage
\subsection{Opgave 3}
Zowel het $x'y'z'$-assenstelsel als het $x''y''z''$-assenstelsel zijn vast aan het vliegtuig gemonteerd. Dit heeft als gevolg dat de relatieve component van de snelheid van het punt C gelijk is met zowel het $x'y'z'$-assenstelsel als het $x''y''z''$-assenstelsel als bewegend assenstelsel. Dit geeft voor de versnelling dezelfde coriolis bijdrage als bij opgave 2.
\begin{equation}
\begin{split}
\vec{a}_{comp}^{'}&=2\cdot(\vec{\omega}_{g}^{'}\times\vec{v}_{rel}^{'})\\
&=2\cdot
\begin{vmatrix}
e_{x}&e_{y}&e_{z}\\
0&0&\omega_{g}\\
\omega_{i}(\sin(\beta)l_{4}-\cos(\beta)l_{3})&0&\omega_{i}(\cos(\beta)l_{4}+\sin(\beta)l_{3})\\
\end{vmatrix}\\
&=\left[ \begin {array}{c} 0\\2\,\omega_{g}\,\omega_
{i}\, \left( \sin \left( \beta \right) l_{4}-\cos \left( \beta
 \right) l_{3} \right) \\ 0\end {array} \right]
\end{split}
\label{eq:kin3.1}
\end{equation}
\subsection{Opgave 4}
\subsubsection{Snelheid}
Om de snelheid van het punt D te bepalen zullen we dezelfde werkwijze als bij opgave 2 hanteren (een combinatie van samengestelde beweging en som van rotaties). We beginnen echter met de afstandsvectoren.
\begin{equation}
\vec{r}_{AB}^{'}=
\begin{bmatrix}
l_{1}\\
0\\
l_{2}\\
\end{bmatrix}
\label{eq:kin4.1}
\end{equation}
\begin{equation}
\vec{r}_{BD}^{'}=
\begin{bmatrix}
\frac{3}{4} l_{4}\\
0\\
\frac{1}{4}l_{3}\\
\end{bmatrix}
\label{eq:kin4.2}
\end{equation}
\begin{equation}
\vec{r}_{AD}^{'}=
\begin{bmatrix}
l_{1}+\frac{3}{4}l_{4}\\
0\\
l_{2}+\frac{1}{4}l_{3}\\
\end{bmatrix}
\label{eq:kin4.3}
\end{equation}
De sleepsnelheid uitgedrukt in het $x'y'z'$-assenstelsel.
\begin{equation}
\begin{split}
\vec{v}_{sleep}^{'}&=\vec{v}_{v}^{'}+\vec{\omega}_{g}^{'}\times\vec{r}_{AD}^{'}\\
&=\begin{bmatrix}
0						\\
v_{v}	\\
0						\\
\end{bmatrix}
+\begin{vmatrix}
e_{x}&e_{y}&e_{z}\\
0&0&\omega_{g}\\
l_{1}+\frac{3}{4}l_{4}&0&l_{2}+\frac{1}{4}l_{3}\\
\end{vmatrix}\\
&=\begin{bmatrix}
0						\\
v_{v}+\omega_{g}\,\left( l_{1}+\frac{3}{4} l_{4}\right)\\
0						\\
\end{bmatrix}
\end{split}
\label{eq:kin4.4}
\end{equation}
Ook de relatieve snelheid wordt in het $x'y'z'$-assenstelsel uitgedrukt.
\begin{equation}
\begin{split}
\vec{v}_{rel}^{'}&=\vec{\omega}_{i}^{'}\times\vec{r}_{BD}^{'}\\
&=\begin{vmatrix}
e_{x}&e_{y}&e_{z}\\
0&\omega_{i}&0\\
\frac{3}{4}l_{4}&0&\frac{1}{4}l_{3}\\
\end{vmatrix}\\
&=\left[\begin{array}{c}\frac{1}{4}\,\omega_{i}\,l_{3}\\0\\-\frac{3}{4}\,\omega_{i}\,l_{4}\end{array}\right]
\end{split}
\label{eq:kin4.5}
\end{equation}
Na optelling van de sleepsnelheid en relatieve snelheid bekomen we \eqref{eq:kin4.6}.
\begin{equation}
\vec{v}_{d}^{'}=\left[ \begin {array}{c} 1/4\,\omega_{i}\,l_{3}\\ \noalign{\medskip}v_{v}+\omega_{g}\, \left( l_{1}+3/4\,l_{4} \right) \\ \noalign{\medskip}-3/4\,\omega_{i}\,l_{4}\end {array} \right]
\label{eq:kin4.6}
\end{equation}
\subsubsection{Versnelling}
De versnelling van het punt D wordt op volledig dezelfde manier bepaald als de versnelling van het punt C als in \ref{sec:versnelling}. Hieronder worden enkel de formules met hun uitwerking neergeschreven vermits de werkwijze eerder al werd verduidelijkt.
\begin{equation}
\begin{split}
\vec{a}_{sleep}^{'}&=\vec{a}_{v}^{'}+\vec{\alpha}_{g}^{'}\times\vec{r}_{AD}^{'}+\vec{\omega}_{g}^{'}\times(\vec{v}_{d}^{'}-\vec{v}_{a}^{'})\\
&=\begin{bmatrix}
0						\\
\alpha_{v}	\\
0						\\
\end{bmatrix}
+\begin{vmatrix}
e_{x}&e_{y}&e_{z}\\
0&0&\alpha_{g}\\
l_{1}+\frac{3}{4}l_{4}&0&l_{2}+\frac{1}{4}l_{3}\\
\end{vmatrix}\\
&+\begin{vmatrix}
e_{x}&e_{y}&e_{z}\\
0&0&\omega_{g}\\
\frac{1}{4}\,\omega_{i}\,l_{3}&\omega_{g}\, \left( l_{1}+\frac{3}{4}\,l_{4} \right)&-\frac{3}{4}\,\omega_{i}\,l_{4}\\
\end{vmatrix}\\
&=\left[ \begin {array}{c} -{\omega_{g}}^{2} \left( l_{1}+\frac{3}{4}\,l_{4} \right) \\a_{v}+\alpha_{g}\, \left( l_{1}+\frac{3}{4}\,l_{4} \right) +\frac{1}{4}\,\omega_{g}\,\omega_{i}\,l_{3}\\0\end {array} \right]
\label{eq:kin4.7}
\end{split}
\end{equation}
\begin{equation}
\begin{split}
\vec{a}_{rel}^{'}&=\vec{\alpha}_{i}^{'}\times\vec{r}_{BD}^{'}+\vec{\omega}_{i}^{'}\times(\vec{v}_{d}^{'}-\vec{v}_{b}^{'})\\
&=\begin{vmatrix}
e_{x}&e_{y}&e_{z}\\
0&\alpha_{i}&0\\
\frac{3}{4}l_{4}&0&\frac{1}{4}l_{3}\\
\end{vmatrix}\\
&+\begin{vmatrix}
e_{x}&e_{y}&e_{z}\\
0&\omega_{i}&0\\
\frac{1}{4}\,\omega_{i}\,l_{3}&\omega_{g}\, \left( l_{1}+\frac{3}{4}\,l_{4} \right)-l_{1}\,\omega_{g}&-\frac{3}{4}\,\omega_{i}\,l_{4}\\
\end{vmatrix}\\
&=\left[ \begin {array}{c} 1/4\,\alpha_{i}\,l_{3}-3/4\,{\omega_{i}}^{2}l_{4}\\0\\-3/4\,\alpha_{i}\,l_{4}-1/4\,{\omega_{i}}^{2}l_{3}\end {array} \right]
\end{split}
\label{eq:kin4.8}
\end{equation}
\begin{equation}
\begin{split}
\vec{a}_{comp}^{'}&=2\cdot(\vec{\omega}_{g}^{'}\times\vec{v}_{rel}^{'})\\
&=2\cdot
\begin{vmatrix}
e_{x}&e_{y}&e_{z}\\
0&0&\omega_{g}\\
\frac{1}{4}\,\omega_{i}\,l_{3}\\0\\-\frac{3}{4}\,\omega_{i}\,l_{4}\\
\end{vmatrix}\\
&=\left[ \begin {array}{c} 0\\ \frac{1}{2}\,\omega_{g}\,
\omega_{i}\,l_{3}\\ 0\end {array} \right]
\end{split}
\label{eq:kin4.9}
\end{equation}
\begin{equation}
\vec{v}_{d}^{'}-\vec{v}_{a}^{'}=\vec{v}_{d}^{'}-\vec{v}_{v}^{'}=
\left[ \begin {array}{c} \frac{1}{4}\,\omega_{i}\,l_{3}\\ \noalign{\medskip}\omega_{g}\, \left( l_{1}+\frac{3}{4}\,l_{4} \right) \\ \noalign{\medskip}-\frac{3}{4}\,\omega_{i}\,l_{4}\end {array} \right]
\label{eq:kin4.10}
\end{equation}
\begin{equation}
\begin{split}
\vec{v}_{b}^{'}&=\vec{v}_{v}^{'}+\vec{\omega}_{g}^{'}\times\vec{r}_{AB}^{'}\\
&=\begin{bmatrix}
0\\
v_{v}\\
0\\
\end{bmatrix}
+\begin{vmatrix}
e_{x}&e_{y}&e_{z}\\
0&0&\omega_{g}\\
l_{1}&0&l_{2}\\
\end{vmatrix}\\
&=\begin{bmatrix}
0\\
v_{v}+l_{1}\omega_{g}\\
0\\
\end{bmatrix}
\label{eq:kin4.11}
\end{split}
\end{equation}
De uiteindelijke oplossing:
\begin{equation}
\vec{a}_{tot}^{'}=\left[ \begin {array}{c} -{\omega_{g}}^{2} \left( l_{1}+\frac{3}{4}\,l_{4} \right) +\frac{1}{4} \,\alpha_{i}\,l_{3}-\frac{3}{4}\,{\omega_{i}}^{2}l_{4}\\ \noalign{\medskip}a_{v}+\alpha_{g}\, \left( l_{1}+\frac{3}{4}\,l_{4} \right) +\frac{3}{4}\,\omega_{g}\,\omega_{i}\,l_{3}\\ \noalign{\medskip}-\frac{3}{4}\,\alpha_{i}\,l_{4}-\frac{1}{4}\,{\omega_{i}}^{2}l_{3}\end {array} \right]
\label{eq:kin4.12}
\end{equation}
\begin{equation}
\vec{\alpha}_{tot}=R^{x'y'z' \rightarrow xyz}\cdot\vec{\alpha}_{tot}^{'}
\end{equation}
\newpage