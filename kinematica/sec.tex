e\section{Kinematica}
\subsection{Opgave 1}
\subsubsection{Rotatiesnelheid}
Om de ogenblikkelijke totale rotatiesnelheidsvector $\vec{\omega}_{tot}$ te berekenen zetten we eerst alle rotatiesnelheidsvectoren om naar het $x'y'z'$-assenstelsel. $\vec{\omega}_{g}$ en $\vec{\omega}_{i}$ staan hier al in, dus we moeten enkel $\vec{\omega}_{w}$ nog omzetten. 
\begin{equation}
\begin{split}
\vec{\omega}_{w}^{'}&=R^{x'''y'''z''' \rightarrow x'y'z'} \cdot \vec{\omega}_{w}^{'''}\\
&=\begin{bmatrix}
\cos(\beta)	&			0			&\sin(\beta)\\
0						&			1			&			0		 \\
-\sin(\beta)&			0			&\cos(\beta)\\
\end{bmatrix}
\cdot
\begin{bmatrix}
-\omega_{w}	\\
0						\\
0						\\
\end{bmatrix}\\
&=
\begin{bmatrix}
-\omega_{w}\cdot \cos(\beta)	\\
0						\\
\omega_{w}\cdot \sin(\beta)	\\
\end{bmatrix}
\end{split}
\label{eq:kin1.1}
\end{equation}
Als we al deze vectoren optellen krijgen we $\vec{\omega}_{tot}^{'}$\footnote{Notatie: het aantal accenten als superscript bij vectoren duidt op het assenstelsel waar de vector in is uitgedrukt.}
\begin{equation}
\begin{split}
\vec{\omega}_{tot}^{'}&=\vec{\omega}_{g}^{'} + \vec{\omega}_{i}^{'} + \vec{\omega}_{w}^{'}\\
&=\begin{bmatrix}
0						\\
0						\\
\omega_{g}	\\
\end{bmatrix}
+\begin{bmatrix}
0						\\
\omega_{i}	\\
0						\\
\end{bmatrix}
+\begin{bmatrix}
-\omega_{w}\cdot \cos(\beta)	\\
0						\\
\omega_{w}\cdot \sin(\beta)	\\
\end{bmatrix}\\
&=\begin{bmatrix}
-\omega_{w}\cdot \cos(\beta)	\\
\omega_{i}						\\
\omega_{g}+\omega_{w}\cdot \sin(\beta)	\\
\end{bmatrix}
\end{split}
\label{eq:kin1.2}
\end{equation}
Na omvorming naar het wereldassenstelsel krijgen we de ogenblikkelijke totale rotatiesnelheidsvector $\vec{\omega}_{tot}$.
\begin{equation}
\begin{split}
\vec{\omega}_{tot}&=R^{x'y'z' \rightarrow xyz} \cdot \vec{\omega}_{tot}^{'}\\
&=\begin{bmatrix}
1			&			0			&			0		   \\
0			&\cos(\alpha)&-\sin(\alpha)\\
0			&\sin(\alpha)&\cos(\alpha) \\
\end{bmatrix}
\cdot
\begin{bmatrix}
-\omega_{w}\cdot \cos(\beta)	\\
0						\\
\omega_{g}+\omega_{i}+\omega_{w}\cdot \sin(\beta)	\\
\end{bmatrix}\\
&=
\begin{bmatrix}
-\omega_{w}\cdot \cos(\beta)	\\
-(\omega_{g}+\omega_{i}+\omega_{w}\cdot \sin(\beta))	\cdot \sin(\alpha)			\\
(\omega_{g}+\omega_{i}+\omega_{w}\cdot \sin(\beta))	\cdot \cos(\alpha)	\\
\end{bmatrix}
\end{split}
\label{eq:kin1.3}
\end{equation}
\newpage
\subsubsection{Rotatieversnelling}
Om de ogenblikkelijke totale rotatieversnellingsvector $\vec{\alpha}_{tot}$ te berekenen maken we gebruik van formule \eqref{eq:kin1.4}.
\begin{equation}
\vec{\alpha} = \sum_{i=1}^{k}{\frac{\mathrm{d}\omega_{i}}{\mathrm{d}t}\vec{e}_{\omega_{i}}}+\sum_{i=1}^{k}{\vec{\omega}_{i}\frac{\mathrm{d}\vec{e}_{\omega_{i}}}{\mathrm{d}t}}  
\label{eq:kin1.4}
\end{equation}
met:
\begin{equation}
\vec{\omega}_{i}\frac{\mathrm{d}\vec{e}_{\omega_{i}}}{\mathrm{d}t}=\sum_{j=1}^{i-1}{\vec{\omega}_{j}\times\vec{\omega}_{i}}  
\label{eq:kin1.5}
\end{equation}
Concreet wordt dit:
\begin{equation}
\vec{\alpha}_{tot} = \vec{\alpha}_{g} + \vec{\alpha}_{i} + \vec{\alpha}_{w} + \vec{\omega}_{g} \times \vec{\omega}_{i} + \vec{\omega}_{g} \times \vec{\omega}_{w} + \vec{\omega}_{i} \times \vec{\omega}_{w}
\label{eq:kin1.6}
\end{equation}
$\vec{\omega}_{g} \times \vec{\omega}_{w}$ en $\vec{\omega}_{i} \times \vec{\omega}_{w}$ zijn termen die duiden op de verandering van oriëntatie van $\vec{\omega}_{w}$. Zijn oriëntatie in namelijk is afhankelijk van $\vec{\omega}_{g}$ en $\vec{\omega}_{i}$. $\vec{\omega}_{g} \times \vec{\omega}_{i}$ is om dezelfde reden toegevoegd.

$\vec{\alpha}_{g}$, $\vec{\alpha}_{i}$, $\vec{\omega}_{g}$ en $\vec{\omega}_{i}$ zijn reeds uitgedrukt in het $x'y'z'$-assenstelsel. $\vec{\omega}_{w}$ werd al in \eqref{eq:kin1.1} naar dit assenstelsel getransformeerd. Hieronder wordt $\vec{\alpha}_{w}$ getransformeerd.
\begin{equation}
\begin{split}
\vec{\alpha}_{w}^{'}&=R^{x'''y'''z''' \rightarrow x'y'z'} \cdot \vec{\alpha}_{w}^{'''}\\
&=\begin{bmatrix}
\cos(\beta)	&			0			&\sin(\beta)\\
0						&			1			&			0		 \\
-\sin(\beta)&			0			&\cos(\beta)\\
\end{bmatrix}
\cdot
\begin{bmatrix}
-\alpha_{w}	\\
0						\\
0						\\
\end{bmatrix}\\
&=
\begin{bmatrix}
-\alpha_{w}\cdot \cos(\beta)	\\
0						\\
\alpha_{w}\cdot \sin(\beta)	\\
\end{bmatrix}
\end{split}
\label{eq:kin1.7}
\end{equation}
Uitgewerkt wordt $\vec{\alpha}_{tot}$:
\begin{equation}
\begin{split}
\vec{\alpha}_{tot}^{'} &= \vec{\alpha}_{g}^{'} + \vec{\alpha}_{i}^{'} + \vec{\alpha}_{w}^{'} + \vec{\omega}_{g}^{'} \times \vec{\omega}_{i}^{'} + \vec{\omega}_{g}^{'} \times \vec{\omega}_{w}^{'} + \vec{\omega}_{i}^{'} \times \vec{\omega}_{w}^{'}\\
&=\begin{bmatrix}
0						\\
0						\\
\alpha_{g}	\\
\end{bmatrix}
+\begin{bmatrix}
0						\\
\alpha_{i}	\\
0						\\
\end{bmatrix}
+\begin{bmatrix}
-\alpha_{w}\cdot \cos(\beta)	\\
0						\\
\alpha_{w}\cdot \sin(\beta)	\\
\end{bmatrix}
+\begin{vmatrix}
e_{x}&e_{y}&e_{z}\\
0&0&\omega_{g}\\
0&\omega_{i}&0\\
\end{vmatrix}\\
&+\begin{vmatrix}
e_{x}&e_{y}&e_{z}\\
0&0&\omega_{g}\\
-\omega_{w}\cdot \cos(\beta)&0&\omega_{w}\cdot \sin(\beta)\\
\end{vmatrix}
+\begin{vmatrix}
e_{x}&e_{y}&e_{z}\\
0&\omega_{i}&0\\
-\omega_{w}\cdot \cos(\beta)&0&\omega_{w}\cdot \sin(\beta)\\
\end{vmatrix}\\
&=\begin{bmatrix}
-\alpha_{w}\cos(\beta)-\omega_{i}\omega_{g}+\omega_{i}\omega_{w}\sin{\beta}\\
\alpha_{i}-\omega_{g}\omega_{w}\cos{\beta}\\
\alpha_{g}+	\alpha_{w} \sin(\beta)+\omega_{i}\omega_{w}\cos{\beta}\\
\end{bmatrix}
\label{eq:kin1.8}
\end{split}
\end{equation}
\begin{equation}
\begin{split}
\vec{\alpha}_{tot}&=R^{x'y'z' \rightarrow xyz} \cdot \vec{\alpha}_{tot}^{'}\\
&=\begin{bmatrix}
1			&			0			&			0		   \\
0			&\cos(\alpha)&-\sin(\alpha)\\
0			&\sin(\alpha)&\cos(\alpha) \\
\end{bmatrix}
\cdot
\begin{bmatrix}
-\alpha_{w}\cos(\beta)-\omega_{i}\omega_{g}+\omega_{i}\omega_{w}\sin{\beta}\\
\alpha_{i}-\omega_{g}\omega_{w}\cos{\beta}\\
\alpha_{g}+	\alpha_{w} \sin(\beta)+\omega_{i}\omega_{w}\cos{\beta}\\
\end{bmatrix}
\end{split}
\label{eq:kin1.9}
\end{equation}
De uitwerking wordt hier achterwege gelaten wegens te complex.
\newpage
\subsection{Opgave 2}
\subsubsection{Snelheid}
Om de ogenblikkelijke snelheid $\vec{v}_{c}$ van het punt C te berekenen hebben wij voor een combinatie van methodes gekozen. Ten eerste is er een samengestelde beweging met het $x'y'z'$-assenstelsel als bewegend assenstelsel. Hierdoor heeft de sleepsnelheid in \eqref{eq:kin2.6} een component veroorzaakt door de translatie van het vliegtuig met snelheid $\vec{v}_{v}$ en een component veroorzaakt door de rotatie van het vliegtuig met rotatiesnelheid $\vec{\omega}_{g}$. Ten tweede maken we gebruik van de methode van som van rotaties om de relatieve snelheid in \eqref{eq:kin2.7} van de eerder vermelde samengestelde beweging te berekenen. We moeten eerst echter de afstandsvectoren berekenen.
\begin{equation}
\vec{r}_{AB}^{'}=
\begin{bmatrix}
l_{1}\\
0\\
l_{2}\\
\end{bmatrix}
\label{eq:kin2.3}
\end{equation}
\begin{equation}
\vec{r}_{BC}^{'}=
\begin{bmatrix}
-(l_{3}\sin{\beta}+l_{4}\cos{\beta})\\
0\\
-l_{3}\cos{\beta}+l_{4}\sin{\beta}\\
\end{bmatrix}
\label{eq:kin2.4}
\end{equation}
\begin{equation}
\vec{r}_{AC}^{'}=
\begin{bmatrix}
l_{1}-(l_{3}\sin{\beta}+l_{4}\cos{\beta})\\
0\\
l_{2}-l_{3}\cos{\beta}+l_{4}\sin{\beta}\\
\end{bmatrix}
\label{eq:kin2.5}
\end{equation}
De sleepsnelheid uitgedrukt in het $x'y'z'$-assenstelsel.
\begin{equation}
\begin{split}
\vec{v}_{sleep}^{'}&=\vec{v}_{v}^{'}+\vec{\omega}_{g}^{'}\times\vec{r}_{AC}^{'}\\
&=\begin{bmatrix}
0						\\
\vec{v}_{v}	\\
0						\\
\end{bmatrix}
+\begin{vmatrix}
e_{x}&e_{y}&e_{z}\\
0&0&\omega_{g}\\
l_{1}-(l_{3}\sin{\beta}+l_{4}\cos{\beta})&0&l_{2}-l_{3}\cos{\beta}+l_{4}\sin{\beta}\\
\end{vmatrix}\\
\end{split}
\label{eq:kin2.6}
\end{equation}
Ook de relatieve snelheid wordt in het $x'y'z'$-assenstelsel uitgedrukt. $\vec{\omega}_{w}$ is niet opgenomen in de formule omdat deze geen bijdrage bij de snelheid van C levert.
\begin{equation}
\begin{split}
\vec{v}_{rel}^{'}&=\vec{\omega}_{i}^{'}\times\vec{r}_{BC}^{'}\\
&=\begin{vmatrix}
e_{x}&e_{y}&e_{z}\\
0&\omega_{i}&0\\
-(l_{3}\sin{\beta}+l_{4}\cos{\beta})&0&-l_{3}\cos{\beta}+l_{4}\sin{\beta}\\
\end{vmatrix}\\
\end{split}
\label{eq:kin2.7}
\end{equation}
\subsubsection{Versnelling}
\subsection{Opgave 3}
\subsection{Opgave 4}